\documentclass[a4paper]{article}
\usepackage[utf8]{inputenc} % Pacote para acentuação
\usepackage[brazil]{babel} % Para quem vai escrever em português brasileiro
\usepackage[lmargin=3cm,tmargin=3cm,rmargin=2cm,bmargin=2cm]{geometry} % Formato que lembra a ABNT
\usepackage[T1]{fontenc} %Ajusta o texto que vem de outras fontes
\usepackage{amsmath,amsthm,amsfonts,amssymb,dsfont,mathtools} % Pacotes matemáticos
\usepackage{blindtext}


\begin{document}
\maketitle{}
\section{O Universo dos Jogos}
Desde os primórdios da humanidade, os jogos são uma parte fundamental da nossa cultura e uma das formas mais puras de entretenimento e aprendizado. Eles são muito mais que simples passatempos; são ferramentas poderosas para estimular a criatividade, o raciocínio lógico, a socialização e até mesmo a resiliência. Seja em um tabuleiro, um campo ou uma tela, os jogos criam universos com regras próprias, nos convidando a assumir papéis, superar obstáculos e viver experiências que, muitas vezes, transcendem o mundo real. Eles são uma linguagem universal, capaz de conectar pessoas de todas as idades e origens através da diversão e do desafio compartilhado.

\textbf{
Esta natureza dual do jogo — lúdica e séria — é o que o torna tão poderoso. Ele opera em um "espaço mágico", um círculo delineado por regras e consentimento mútuo, onde ações assumem um significado especial e temporário. Dentro desse espaço, a derrota é suportável e o risco, fictício, o que nos permite experimentar, falhar e tentar novamente sem as consequências graves do mundo real. É um simulador de experiências, um campo de treino para a resiliência, o pensamento crítico, a cooperação e a competição saudável.}


\begin{center}
\subsection{Os Jogos Online}
\underline{
Os jogos online representam a evolução digital dessa antiga forma de conexão.} Eles quebram as barreiras geográficas, criando praças virtuais onde milhões de jogadores podem interger, cooperar ou competir em tempo real, independentemente de onde estejam. Mais do que gráficos avançados ou mechanics complexas, a essência desses jogos reside na comunidade que os habita. Mundos persistentes, batalhas épicas coordenadas por voz e amizades forjadas em partidas casuais transformam a experiência em algo social e dinâmico, fazendo do jogo não apenas um universo para se explorar, mas um lar digital para se pertencer.
\begin{end}

\subsection{Os Jogos Físicos}
\begin{flushleft}    
\textit{
Por outro lado, os jogos físicos preservam a magia tangível e a convivência presencial. Segurar as cartas de um baralho, mover peças em um tabuleiro, ou sentir a textura de uma bola oferecem uma experiência sensorial única e insubstituível. Eles são catalisadores de interação face a face, promovendo momentos de risada, estratégia compartilhada e olho no olho que a tela não pode replicar. Seja em um jogo de tabuleiro familiar, uma partida de futebol no parque ou um quebra-cabeça montado em grupo, esses jogos reforçam laços no mundo real, celebrando o tato, a presença e a simplicidade de estar junto, offline.}
\end{flushleft}


\end{document}