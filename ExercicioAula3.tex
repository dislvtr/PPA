\documentclass[a4paper]{article}
\usepackage[utf8]{inputenc} % Pacote para acentuação
\usepackage[brazil]{babel} % Para quem vai escrever em português brasileiro
\usepackage[lmargin=3cm,tmargin=3cm,rmargin=2cm,bmargin=2cm]{geometry} % Formato que lembra a ABNT
\usepackage[T1]{fontenc} %Ajusta o texto que vem de outras fontes
\usepackage{amsmath,amsthm,amsfonts,amssymb,dsfont,mathtools} % Pacotes matemáticos
\usepackage{blindtext}
\usepackage{graphicx}


\begin{document}

\maketitle{}

\begin{figure}[ht]
\centering
\includegraphics[width=2cm]{marca.png} 
\caption{Logomarca do Brawl Stars}
\label{Fig01}
\end{figure}

Eu amei a logo marca do Brawl Stars que é a figura \ref{Fig01}

\begin{figure}
    \centering
    \includegraphics[width=5cm]{imagem.png}
    \caption{Logomarca do LOL}
    \label{Fig02}
\end{figure}

Eu gostei mais da logomarca \ref{Fig02} do que da \ref{Fig01}

\end{document}